%! Author = Wojciech Kot - 151879
% Preamble
\documentclass[11pt]{article}

% Packages
\usepackage{amsmath}
\usepackage[a4paper, margin=0.5in]{geometry}
\usepackage{graphicx} % daj an 1in jak chcesz normalniejszy margines, ale kod mi się w linii nie mieści :P
\usepackage[utf8]{inputenc}
\usepackage[T1]{fontenc}
\usepackage[polish]{babel}
\usepackage{float}
\usepackage{hyperref}
\usepackage{cleveref}
% \usepackage{subfigure}

\title{Raport ze studium przypadku}
\author{Wojciech Kot 151879}
\date{}

\begin{document}

\maketitle
\newpage

\section{Wstęp}\label{sec:wstep}

Zbiór danych, który zdecydowałem się przebadać to zbiór dostępny pod linkiem:
\href{https://www.kaggle.com/datasets/anandshaw2001/video-game-sales}{video-game-sales}.
Pochodzi on z serwisu kaggle i zawiera on informacje dotyczące sprzedaży gier wideo na różnych platformach sprzętowych,
w różnych regionach świata i w rozmaitych gatunkach gier.
Dane te mogą być wykorzystane zarówno do analizy trendów rynkowych,
jak i do przewidywania sprzedaży w zależności od cech gry, w tym takich jak jej wydawca, platforma, czy rok wydania.
Sam jestem ciekawy powiązań jakie (możliwe, że) uda się odkryć,
chociażby korelacji wydawcy ze sprzedażą i pewnych różnic pomiędzy wydawcami a sprzedażą w konkretnych rejonach świata.

\section{Opis zbioru danych}\label{sec:opis-zbioru-danych}

Zbiór danych składa się z \textbf{11 kolumn} i \textbf{16\,598 rekordów}.
Każdy rekord odpowiada jednej grze wideo na konkretnej platformie — oznacza to,
że jeśli dana gra została wydana na kilku platformach (np.\ PC i PS4),
to pojawia się w zbiorze danych wielokrotnie, jako osobne wpisy dla każdej z tych platform.

Każdy rekord zawiera następujące atrybuty:
\begin{itemize}
  \item \textbf{Rank} – pozycja gry w globalnym rankingu sprzedaży,
  \item \textbf{Name} – tytuł gry,
  \item \textbf{Platform} – platforma, na której gra została wydana (np. PC, PS4),
  \item \textbf{Year} – rok wydania gry,
  \item \textbf{Genre} – gatunek gry (np. Action, Sports),
  \item \textbf{Publisher} – wydawca gry,
  \item \textbf{NA\_Sales} – sprzedaż w Ameryce Północnej (w milionach egzemplarzy),
  \item \textbf{EU\_Sales} – sprzedaż w Europie (w milionach egzemplarzy),
  \item \textbf{JP\_Sales} – sprzedaż w Japonii (w milionach egzemplarzy),
  \item \textbf{Other\_Sales} – sprzedaż w pozostałych regionach świata (w milionach egzemplarzy),
  \item \textbf{Global\_Sales} – łączna sprzedaż na całym świecie (w milionach egzemplarzy).
\end{itemize}

Zbiór zawiera również brakujące wartości. \\
\textbf{Publisher} ma 58 brakujących wartości, oraz \\
\textbf{Year} ma 271 brakujących wartości.

\section{Eksploracja danych}\label{sec:eksploracja-danych}

\subsection{Histogramy zmiennych ilościowych}\label{subsec:histogramy-zmiennych-ilosciowych}

Dla wszystkich zmiennych ilościowych (w tym sprzedaży regionalnej i globalnej) wygenerowałem histogramy.
Analiza wykazała silne skośności w rozkładach -- zdecydowana większość gier osiąga bardzo niską sprzedaż (poniżej 0.5 miliona egzemplarzy),
natomiast tylko nieliczne tytuły mają sprzedaż przekraczającą 5 milionów.
Sugeruje to obecność efektu długiego ogona typowego dla rynku gier, jak i dla ogółu rynków produktów rozrywkowych.

\vspace{0.3cm}
\noindent
\textit{(TODO: W tym miejscu można umieścić ilustrację z histogramami)}
\vspace{0.3cm}

\subsection{Analiza korelacji}\label{subsec:analiza-korelacji}

Obliczyłem macierz korelacji pomiędzy zmiennymi ilościowymi i przedstawiam ją graficznie jako heatmapę.
Najsilniejsze korelacje zaobserwowano między regionalnymi sprzedażami a \texttt{Global\_Sales},
co jest zgodne z intuicją – wartości te są składnikami łącznej sprzedaży.
W szczególności:
\begin{itemize}
  \item \texttt{NA\_Sales} i \texttt{Global\_Sales}: wysoka dodatnia korelacja,
  \item \texttt{EU\_Sales} i \texttt{Global\_Sales}: wysoka dodatnia korelacja,
  \item \texttt{JP\_Sales} – znacznie niższa stosunkowo korelacja z resztą regionów, co może wskazywać na odrębność rynku japońskiego.
\end{itemize}

Po obliczeniu korelacji, należy zwrócić uwagę na rynek japoński.
Wygląda na to, że nie podąża on do końca za światowym trendem,
a niska korelacja sugeruje, że mogą panować na nim zupełnie inne trendy.

\vspace{0.3cm}
\noindent
\textit{(TODO: W tym miejscu można wstawić wykres z macierzą korelacji – heatmapę)}
\vspace{0.3cm}

\subsection{Sprzedaż w czasie}\label{subsec:sprzedaz-w-czasie}

Zbadałem również rozkład sprzedaży gier w kolejnych latach.
Dane pokazują wzrost liczby wydawanych gier oraz ich sprzedaży począwszy od lat 90.,
z wyraźnym szczytem w okolicach roku 2008, po którym obserwowany jest stopniowy spadek.
Spadek ten może wynikać zarówno z faktycznego zmniejszenia liczby wydań, jak i niepełności danych w późniejszych latach.
Z własnej wiedzy jestem w stanie stwierdzić, że głównym powodem tego spadku będą brakujące dane,
szczególnie braki w latach 2020 i późniejszych, kiedy rynek gier miał swego rodzaju ``złotą erę''.

\vspace{0.3cm}
\noindent
\textit{(TODO: W tym miejscu można wstawić wykres liniowy przedstawiający sumę sprzedaży lub liczbę gier na rok)}
\vspace{0.3cm}

\subsection{TODO}\label{subsec:todo}
No uzupełnić dalej :)



\section{Wyniki predykcji modeli}\label{sec:wyniki-predykcji-modeli}


\section{Link do repozytorium}\label{sec:link-do-repo}
Kod źródłowy w repozytorium GitHub dostępny pod linkiem: \\
\href{https://github.com/KotZPolibudy/PUT_SUS/zdataset-analiza}{Repozytorium}.

\end{document}
