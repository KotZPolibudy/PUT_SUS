%! Author = Wojciech Kot - 151879
% Preamble
\documentclass[11pt]{article}

% Packages
\usepackage{amsmath}
\usepackage[a4paper, margin=0.5in]{geometry}
\usepackage{graphicx} % daj an 1in jak chcesz normalniejszy margines, ale kod mi się w linii nie mieści :P
\usepackage[utf8]{inputenc}
\usepackage[T1]{fontenc}
\usepackage[polish]{babel}
\usepackage{float}
\usepackage{hyperref}
\usepackage{cleveref}
\usepackage{subfigure}

\title{Zadanie 5. Hybrydowy Algorytm Ewolucyjny}
\author{Wojciech Kot 151879}
\date{}

\begin{document}

\maketitle
\newpage

\section{Opis zbioru danych}\label{sec:opis-zbioru-danych}

Zbiór danych, który zdecydowałem się przebadać to zbiór dostępny pod linkiem:
\href{https://www.kaggle.com/datasets/anandshaw2001/video-game-sales}{video-game-sales}.
Pochodzi on z serwisu kaggle i zawiera on informacje dotyczące sprzedaży gier wideo na różnych platformach sprzętowych,
w różnych regionach świata i w rozmaitych gatunkach gier.
Dane te mogą być wykorzystane zarówno do analizy trendów rynkowych,
jak i do przewidywania sprzedaży w zależności od cech gry, w tym takich jak jej wydawca, platforma, czy rok wydania.

Zbiór danych składa się z \textbf{11 kolumn} i \textbf{16\,598 rekordów}.
Każdy rekord odpowiada jednej grze wideo na konkretnej platformie — oznacza to,
że jeśli dana gra została wydana na kilku platformach (np.\ PC i PS4),
to pojawia się w zbiorze danych wielokrotnie, jako osobne wpisy dla każdej z tych platform.

Każdy rekord zawiera następujące atrybuty:
\begin{itemize}
  \item \textbf{Rank} – pozycja gry w globalnym rankingu sprzedaży,
  \item \textbf{Name} – tytuł gry,
  \item \textbf{Platform} – platforma, na której gra została wydana (np. PC, PS4),
  \item \textbf{Year} – rok wydania gry,
  \item \textbf{Genre} – gatunek gry (np. Action, Sports),
  \item \textbf{Publisher} – wydawca gry,
  \item \textbf{NA\_Sales} – sprzedaż w Ameryce Północnej (w milionach egzemplarzy),
  \item \textbf{EU\_Sales} – sprzedaż w Europie (w milionach egzemplarzy),
  \item \textbf{JP\_Sales} – sprzedaż w Japonii (w milionach egzemplarzy),
  \item \textbf{Other\_Sales} – sprzedaż w pozostałych regionach świata (w milionach egzemplarzy),
  \item \textbf{Global\_Sales} – łączna sprzedaż na całym świecie (w milionach egzemplarzy).
\end{itemize}

Zbiór zawiera również brakujące wartości.
\textbf{Publisher} ma 58 brakujących wartości, oraz
\textbf{Year} ma 271 brakujących wartości.

\section{Eksploracja danych}\label{sec:eksploracja-danych}

\subsection{Najczęściej występujące rekordy}\label{subsec:najczesciej-wystepujace-rekordy}


\section{Wyniki predykcji modeli}\label{sec:wyniki-predykcji-modeli}


\section{Link do repozytorium}\label{sec:link-do-repo}
Kod źródłowy w repozytorium GitHub dostępny pod linkiem: \\
\href{https://github.com/KotZPolibudy/PUT_SUS/zdataset-analiza}{Repozytorium}.

\end{document}
